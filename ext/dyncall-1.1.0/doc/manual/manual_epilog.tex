%//////////////////////////////////////////////////////////////////////////////
%
% Copyright (c) 2007,2009 Daniel Adler <dadler@uni-goettingen.de>, 
%                         Tassilo Philipp <tphilipp@potion-studios.com>
%
% Permission to use, copy, modify, and distribute this software for any
% purpose with or without fee is hereby granted, provided that the above
% copyright notice and this permission notice appear in all copies.
%
% THE SOFTWARE IS PROVIDED "AS IS" AND THE AUTHOR DISCLAIMS ALL WARRANTIES
% WITH REGARD TO THIS SOFTWARE INCLUDING ALL IMPLIED WARRANTIES OF
% MERCHANTABILITY AND FITNESS. IN NO EVENT SHALL THE AUTHOR BE LIABLE FOR
% ANY SPECIAL, DIRECT, INDIRECT, OR CONSEQUENTIAL DAMAGES OR ANY DAMAGES
% WHATSOEVER RESULTING FROM LOSS OF USE, DATA OR PROFITS, WHETHER IN AN
% ACTION OF CONTRACT, NEGLIGENCE OR OTHER TORTIOUS ACTION, ARISING OUT OF
% OR IN CONNECTION WITH THE USE OR PERFORMANCE OF THIS SOFTWARE.
%
%//////////////////////////////////////////////////////////////////////////////

\newpage

\section{Epilog}

\subsection{Stability and security considerations}

Since the \product{dyncall} library doesn't know anything about the called
function itself (except its address), no parameter-type validation is done.
This means that in order to avoid crashes, data corruption, etc., the user is
required to ascertain the number and types of parameters. It is strongly advised to
double check the parameter types of every function to be called, and not to
call unknown functions at all.\\

Consider a simple program that issues a call by directly passing some
unchecked command line arguments to the call itself, or even worse, by indirectly
choosing a library to load and a function to call without verification.
Such unchecked input data can quite easily be used to intentionally crash the
program or to take over control of the program flow.\\
If not used with care, programs depending on the \product{dyncall},
\product{dyncallback} and \product{dynload} libraries, can be exploited as
arbitrary function call dispatchers through manipulation of their input data.
Successful exploits of badly formed programs like outlined above can be misused
as powerful tools for a wide variety of malicious attacks, \ldots
 

\subsection{Embedding}

The \product{dyncall} library strives to have a minimal set of dependencies,
meaning no required runtime dependencies and usually only the necessary tools
to build the library as build-time dependencies, like a compiler and assembler,
linker, etc..
The library uses some heap-memory to store the CallVM and uses by default the
platform's \capi{malloc()} and \capi{free()} calls. However, providing custom
\capi{dcAllocMem} and \capi{dcFreeMem} C-preprocessor definitions will override
the default behaviour.
See \shell{dyncall/dyncall\_alloc.h} for details.


\subsection{Multi-threading}

The \product{dyncall} library is thread-safe and reentrant, by means that it
works correctly during execution of multiple threads if, and only if there is
at most a single thread pushing arguments to one CallVM. Since there's no
limitation on the number of created CallVM objects, it is recommended to keep a
copy per thread if mutliple threads make use of \product{dyncall} in parallel.
Invoking the call should always be thread-safe, however, whether the called
function is thread-safe is up to the programmer to verify, of course.

\subsection{Supported types}

Currently, the \product{dyncall} library supports all of ANSI C's integer,
floating point and pointer types as function call arguments and return values.
Additionally, C++'s \capi{bool} and C99's \capi{\_Bool} types are supported.
Due to the still rare and often incomplete support of the \capi{long double}
type on various platforms, the latter is currently not officially supported.

\subsection{Roadmap}

The \product{dyncall} library should be extended by a wide variety of other
calling conventions and ported to other and more esoteric platforms. With its
low memory footprint it surely comes in handy on embedded systems.
Furthermore, the authors plan to provide more scripting language bindings,
examples, and other projects based on \product{dyncall}.\\
Besides \product{dyncall} and \product{dyncallback}, the \product{dynload}
library needs to be extended with support for other shared library formats
(e.g. AmigaOS .library or GEM \cite{.ldg} files).


\subsection{Related libraries}

Besides the \product{dyncall} library, there are other free and open projects
with similar goals. The most noteworthy libraries are libffi \cite{libffi},
C/Invoke \cite{cinvoke} and libffcall \cite{libffcall}.

